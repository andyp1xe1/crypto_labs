% Created 2025-09-29 Mon 16:34
% Intended LaTeX compiler: pdflatex
\documentclass[a4paper,12pt]{article}
\usepackage[utf8]{inputenc}
\usepackage[T1]{fontenc}
\usepackage{amsmath}
\usepackage{amssymb}
\usepackage{capt-of}
\usepackage{hyperref}
\usepackage{amsthm}
\usepackage{amssymb}
\usepackage{mathtools}
\input{lab_pre}
\group{FAF--233}
\prof{The Prof}
\profdep{sea, fcim utm}
\labno{5}

%% ox-latex features:
%   !announce-start, !guess-pollyglossia, !guess-babel, !guess-inputenc,
%   maths, !announce-end.

\usepackage{amsmath}
\usepackage{amssymb}

%% end ox-latex features


\author{Andrei Chicu}
\date{\today}
\title{Cifrul lui Cezar: Implementare și Extensie}
\hypersetup{
 pdfauthor={Andrei Chicu},
 pdftitle={Cifrul lui Cezar: Implementare și Extensie},
 pdfkeywords={},
 pdfsubject={},
 pdfcreator={Emacs 30.2 (Org mode 9.8-pre)},
 pdflang={English}}
\begin{document}

\makeatletter
\begin{titlepage}
\centering

\includegraphics[height=2cm]{utm_logo.png}

\bfseries
\textsc{Ministry of Education and Research of Republic of Moldova} \\
\textsc{Technical University of Moldova} \\
\textsc{Faculty of Computers, Informatics and Microelectronics} \\
\textsc{Department of Software and Automation Engineering} \\
\mdseries

\vfill

\textsc{\Large Analysis of Algorithms} \\
\textsc{\large Laboratory work \#\@labno}\\[0.5cm]

\vspace{12pt}
\newcommand{\HRule}{\rule{\linewidth}{0.5mm}}
\HRule \\[0.2cm]
{ \LARGE \bfseries \@title }\\[0.4cm]
\HRule
\vfill

\begin{minipage}[t]{0.4\textwidth}
\begin{flushleft} \large
\emph{Author:} \\
\@author\\                        
std. gr. \@group
\end{flushleft}
\end{minipage}
~
\begin{minipage}[t]{0.4\textwidth}
\raggedleft \large
\emph{Verified:} \\
\@prof \\
Department of \textsc{\@profdep}
\end{minipage}\\[3cm]
\vfill

Chișinău, 2025
\end{titlepage}
\makeatother
\setcounter{page}{2}
\section{Introduction}
\label{sec:org285a67a}
github url: \url{https://github.com/...}
\subsection{Objective}
\label{sec:orgf6ae595}
The objective of this laboratory work is to implement the Caesar Cipher algorithm, covering both the standard fixed-shift version and an extended version that uses a keyword to permute the alphabet, significantly increasing the cipher's key space and cryptoresistance.
\subsection{Tasks}
\label{sec:orgddf626c}
\begin{itemize}
\item Task 1.1: Standard Caesar Cipher (Single Key \(k_{1}\))
\item Task 1.2: Permutation Caesar Cipher (Two Keys \(k_{1}\) and k\(_{2}\)​)
\item Task 1.3: Cipher Verification (Exchange and Decrypt)
\end{itemize}
\subsection{Theoretical Notes}
\label{sec:org790f330}
The standard Caesar cipher uses the formulas:
\begin{itemize}
\item Encryption: \(ck(x)=(x+k)(mod n)\)
\item Decryption: \(mk(y)=(y−k)(mod n)\) where \(n=26\) for the English alphabet and the shift key \(k \in{1,2,…,25}\).
\item The permutation cipher modifies this by using a new alphabet sequence defined \\
by the keyword \(k_{2}\).
\end{itemize}
\subsection{Task 1.1: Standard Caesar Cipher (Single Key)}
\label{sec:orgbc3873b}

\subsubsection{Implementation Details}
\label{sec:org702c33d}
The standard Caesar Cipher implementation uses a single integer key, \(k_{1}\), for the shift.
\begin{enumerate}
\item Key and Text Validation
\label{sec:org536bd7f}

\texttt{getShiftKey}: Validates that the shift key \(k_{1}\) is an integer in the range \([1,25]\).

\texttt{sanitizeText}: Ensures the input plaintext is converted to uppercase and all non-letter characters (including spaces) are removed.
\item Cipher Logic
\label{sec:orgdf83abf}
The core logic resides in the \texttt{processText} function, which uses the constant \texttt{alphabet}
The encryption/decryption is achieved by calculating the index of the new character using modular arithmetic:
\begin{itemize}
\item Find the index of the character x in the standard alphabet.
\item For encryption, calculate \texttt{(index+k)(mod 26)}.
\item For decryption, calculate \texttt{(index-k)(mod 26)}. The result is adjusted to ensure it remains positive (e.g., \texttt{(index-k+26)(mod 26)}).
\end{itemize}
\end{enumerate}
\subsection{Task 1.2: Permutation Caesar Cipher (Two Keys)}
\label{sec:orgf5a0cc6}

\subsubsection{Implementation Details}
\label{sec:org4a46120}
This extended cipher uses a shift key \(k_{1}\) (integer) and a permutation key \(k_{2}\) (keyword string).
\begin{enumerate}
\item Key Validation
\label{sec:org8b29791}
\texttt{getPermutationKey}: Validates the permutation keyword \(k_{2}\):
\begin{itemize}
\item Must be composed only of letters.
\item Must have a minimum length of 7 characters.
\end{itemize}
\item Permuted Alphabet Generation
\label{sec:org431c71d}
The \texttt{generatePermutedAlphabet} function is responsible for creating the new alphabet based on \(k_{2}\):
The keyword \(k_{2}\) is sanitized and uppercased.
Unique letters from \(k_{2}\) are appended to the new alphabet in the order they first appear (duplicates are excluded).
The remaining letters of the standard alphabet (A-Z) that were not in the keyword are appended in their natural order.

\textbf{Example:} For \(k_{2}="cryptography"\), the permuted alphabet is \texttt{CRYPTOGAHBDEFIJKMLNQSUVWXZ}.
\item Cipher Logic
\label{sec:org3018471}
The same \texttt{processText} function is reused, but it now operates on the permuted alphabet string instead of the standard one. The indices are mapped based on the position within this new 26-character sequence.
\end{enumerate}
\subsection{Task 1.3: Cipher Verification}
\label{sec:orgb621f53}

\subsubsection{Implementation Details}
\label{sec:org7369653}
This task verifies the practical application of the Permutation Caesar Cipher through a peer exchange.
\begin{enumerate}
\item Exchange Results
\label{sec:orgceaa041}
\end{enumerate}
\subsection{Conclusions}
\label{sec:org61aee32}
The laboratory work successfully implemented the Caesar Cipher and its extension. The use of a permutation keyword significantly complicates an exhaustive key search compared to the standard version, although the cipher remains vulnerable to frequency analysis. All requirements regarding text sanitization (uppercase, no non-letters) and key validation (\(k_{1}\in[1,25], len(k_{2})\ge7\)) were met in the Go implementation.
\end{document}
